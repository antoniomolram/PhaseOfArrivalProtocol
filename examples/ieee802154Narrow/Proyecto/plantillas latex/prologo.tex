\chapter*{PRÓLOGO}

A la finalización del proyecto nos hemos dado cuenta de que ahora, que ya se ha dado el primer paso en el desarrollo del guia robótico, nos sabe a poco todo el trabajo realizado, porque a partir de este punto es cuando realmente se puede desarrollar una aplicación concreta, y tiene muchas más aplicaciones de las que nos habiamos planteado. Creo que queda patente el entusiasmo por el logro obtenido y la pena por tener que concluir el proyecto en este punto, cuando se pueden hacer tantas cosas.\par
La idea inicial del proyecto fue construir un perro guia robótico para personas con deficiencias visuales. La idea en sí, debido al enfoque, presenta serias complicaciones. No se puede construir un perro electrónico, ni robótico, ni cibernético. La razón es sencilla, cariño, compañía, convivencia\dots \ son cosas que ningún robot, diseñado o por diseñar, puede dar. Son cosas que un perro te da sin pedir nada a cambio. Como bien sabemos todos los que tenemos o hemos tenido mascotas en casa, ninguna máquina puede sustituir a un ser vivo. Al menos esta es mi opinión personal, un perro es un compañero inigualable. Ahora mismo tengo al mío durmiendo aburrido sobre mis pies, convencido de que estoy trabajando y no es hora de jugar.\par
Pero claro, pensando como ingeniero, un robot puede hacer en segundos cosas que nosotros tardaríamos horas, sino días, en hacer con nuestras propias manos. Si nos paramos a pensar un momento\dots \ ?`Cuantas veces, yendo de viaje nos hemos perdido?. Hemos cogido el plano de la ciudad que conseguímos en la oficina de turismo. ?`En que calle nos encontramos? \dots~leemos el cartel de la esquina, la calle tal, miramos el plano y nos orientamos, hay que seguir todo recto por allí. Este sencillo gesto para los que podemos ver se hace imposible para un invidente. Imaginemos por un momento que no podemos ver. Si pudieramos ir acompañados de alguien que nos guiara ya dependeríamos de esa persona durante todo el viaje, además nuestro perro tampoco conoce la ciudad y no nos puede ayudar a orientarnos. Es, en esos momentos, cuando el guía robótico nos daría la libertad de movernos por una ciudad completamente desconocida con la tranquilidad de que no nos vamos a perder. Podríamos programar el recorrido a pie hasta el parque para poder disfrutar del frescor de la sombra, el olor a cesped recien cortado y un ambiente alejado del ruido de los coches, más tarde, cuando tuvieramos hambre, podríamos decirle a nuestro guia particular, nuestro guia robótico, que nos llevara al burguer más cercano, para terminar yendo al hotel. La verdad es que visto así el guia robótico no está tan mal.\par
Mi in\-ten\-ción ha si\-do di\-se\-ñar y cons\-truir las ba\-ses fun\-cio\-na\-les del guia ro\-bó\-ti\-co, centrándome fundamentalmente en la forma de buscar un camino sin obstáculos entre dos posiciones relativamente alejadas. Integrando el sistema GPS (Golbal Positioning System) y dejando abiertas todas las posibilidades de ampliación posibles.\par
